\PassOptionsToPackage{unicode=true}{hyperref} % options for packages loaded elsewhere
\PassOptionsToPackage{hyphens}{url}
%
\documentclass[]{article}
\usepackage{lmodern}
\usepackage{amssymb,amsmath}

% settings
\usepackage{minted}
%



\usepackage{ifxetex,ifluatex}
\usepackage{fixltx2e} % provides \textsubscript
\ifnum 0\ifxetex 1\fi\ifluatex 1\fi=0 % if pdftex
  \usepackage[T1]{fontenc}
  \usepackage[utf8]{inputenc}
  \usepackage{textcomp} % provides euro and other symbols
\else % if luatex or xelatex
  \usepackage{unicode-math}
  \defaultfontfeatures{Ligatures=TeX,Scale=MatchLowercase}
    \setmainfont[]{Source Han Serif CN}
    \setsansfont[]{Source Han Sans CN}
    \setmonofont[Mapping=tex-ansi]{Source Code Pro}
  \ifxetex
    \usepackage{xeCJK}
    \setCJKmainfont[]{Source Han Serif CN}
  \fi
  \ifluatex
    \usepackage[]{luatexja-fontspec}
    \setmainjfont[]{Source Han Serif CN}
  \fi
\fi
% use upquote if available, for straight quotes in verbatim environments
\IfFileExists{upquote.sty}{\usepackage{upquote}}{}
% use microtype if available
\IfFileExists{microtype.sty}{%
\usepackage[]{microtype}
\UseMicrotypeSet[protrusion]{basicmath} % disable protrusion for tt fonts
}{}
\IfFileExists{parskip.sty}{%
\usepackage{parskip}
}{% else
\setlength{\parindent}{0pt}
\setlength{\parskip}{6pt plus 2pt minus 1pt}
}
\usepackage{hyperref}
\hypersetup{
            pdfborder={0 0 0},
            breaklinks=true}
\urlstyle{same}  % don't use monospace font for urls
\usepackage[margin=2cm]{geometry}
\setlength{\emergencystretch}{3em}  % prevent overfull lines
\providecommand{\tightlist}{%
  \setlength{\itemsep}{0pt}\setlength{\parskip}{0pt}}
\setcounter{secnumdepth}{0}
% Redefines (sub)paragraphs to behave more like sections
\ifx\paragraph\undefined\else
\let\oldparagraph\paragraph
\renewcommand{\paragraph}[1]{\oldparagraph{#1}\mbox{}}
\fi
\ifx\subparagraph\undefined\else
\let\oldsubparagraph\subparagraph
\renewcommand{\subparagraph}[1]{\oldsubparagraph{#1}\mbox{}}
\fi


% set default figure placement to htbp
\makeatletter
\def\fps@figure{htbp}
\makeatother

\usepackage{minted}



\date{}

\title{\vspace{50mm} \huge Standard Code Library \\[20pt]}
\author{Your TeamName \\[10pt] Your School}
\date{\today}


\begin{document}

\begin{titlepage}

\maketitle

\end{titlepage}

\newpage

\renewcommand\labelitemi{$\bullet$}

{
\setcounter{tocdepth}{3}
\tableofcontents
\newpage
}








\hypertarget{ux52a8ux6001ux89c4ux5212}{%
\subsection{\# 动态规划}\label{ux52a8ux6001ux89c4ux5212}}

\begin{enumerate}
\def\labelenumi{\arabic{enumi}.}
\tightlist
\item
  背包DP
\end{enumerate}

\begin{minted}[fontsize=\footnotesize,breaklines,linenos]{cpp}
// 多重背包
for (int i = 1; i <= n; i++)
{
    int q = a[i].m;
    for (int j = 1; q; j *= 2)
    {
        if (j > q)
        {
            j = q;
        }
        q -= j;
        for (int k = w; k >= j * a[i].w; k--)
        {
            f[k] = max(f[k], f[k - j * a[i].w] + j * a[i].v);
        }
    }
}
\end{minted}

\hypertarget{ux4e8cux8fdbux5236ux5206ux7ec4ux4f18ux5316}{%
\subsubsection{二进制分组优化}\label{ux4e8cux8fdbux5236ux5206ux7ec4ux4f18ux5316}}

\begin{minted}[fontsize=\footnotesize,breaklines,linenos]{cpp}
index = 0;
for (int i = 1; i <= m; i++)
{
    int c = 1, p, h, k;
    cin >> p >> h >> k;
    while (k > c)
    {
        k -= c;
        list[++index].w = c * p;
        list[index].v = c * h;
        c *= 2;
    }
    list[++index].w = p * k;
    list[index].v = h * k;
}
\end{minted}

\begin{enumerate}
\def\labelenumi{\arabic{enumi}.}
\setcounter{enumi}{1}
\tightlist
\item
  状态压缩DP
\end{enumerate}

\begin{minted}[fontsize=\footnotesize,breaklines,linenos]{cpp}
int cnt[1024];
int dp[40][1024][90];
int can[2000], num = 0;
int S = 1 << n;
for (int s = 0; s < S; s++)
{
    if ((s << 1) & s)
    {
        continue;
    }
    can[++num] = s;
    for (int j = 0; j < n; j++)
    {
        if ((s >> j) & 1)
        {
            cnt[num]++;
        }
    }
    dp[1][num][cnt[num]] = 1;
}
for (int i = 2; i <= n; i++)
{
    for (int j = 1; j <= num; j++)
    {
        int x = can[j];
        for (int p = 1; p <= num; p++)
        {
            int y = can[p];
            if ((y & x) || ((y << 1) & x) || ((y >> 1) & x))
                continue;
            for (int l = 0; l <= k; l++)
            {
                dp[i][j][cnt[j] + l] += dp[i - 1][p][l];
            }
        }
    }
}
\end{minted}

\begin{enumerate}
\def\labelenumi{\arabic{enumi}.}
\setcounter{enumi}{2}
\tightlist
\item
  数位DP
\end{enumerate}

现在问有多少的数比12345小

1.关于前导0:00999→999,09999→9999

2.关于limit前面的数是否紧贴上限

如果前面的数是紧贴上限的,当前这位枚举的上限便是当前数的上限

如果前面的数不是紧贴上限的,当前这位枚举的上限便是 9

3.关于DP维度

一般来说,DFS有几个状态,DP就几个维度  

比如现在DP就是DP {[}pos{]} {[}limt{]} {[}zero{]}

4.关于记忆化DP

现在枚举到了 10××× 和 11×××

显然 这两种状态后面的×××状态数是一样的

重点:dp{[}pos{]}{[}limit{]}{[}zero{]}表示前面的数枚举状态确定,后面的数有多少种可能

5.关于DP细节

一般来说我们一开始都memset(dp,-1,sizeof(dp))

如果dp{[}pos{]}{[}limt{]}{[}zero{]}!=-1 return
dp{[}pos{]}{[}limit{]}{[}zero{]};

6.关于初始化:

一开始 limit 是1,表示一开始的数只能选 1\textasciitilde a{[}1{]}

一开始zero 是1,假定表示前面的数全为0

\begin{minted}[fontsize=\footnotesize,breaklines,linenos]{cpp}
#include <bits/stdc++.h>
using namespace std;
#define ll long long
#define mp make_pair
#define pb push_back  // vector函数
#define popb pop_back // vector函数
#define fi first
#define se second
const int N = 20;
// const int M=;
// const int inf=0x3f3f3f3f;     //一般为int赋最大值,不用于memset中
// const ll INF=0x3ffffffffffff; //一般为ll赋最大值,不用于memset中
int T, n, len, a[N], dp[N][2][2];
inline int read()
{
    int x = 0, f = 1;
    char ch = getchar();
    while (ch < '0' || ch > '9')
    {
        if (ch == '-')
            f = -1;
        ch = getchar();
    }
    while (ch >= '0' && ch <= '9')
    {
        x = (x << 1) + (x << 3) + (ch ^ 48);
        ch = getchar();
    }
    return x * f;
}
int dfs(int pos, bool lim, bool zero)
{
    if (pos > len)
        return 1;
    if (dp[pos][lim][zero] != -1)
        return dp[pos][lim][zero];
    int res = 0, num = lim ? a[pos] : 9;
    for (int i = 0; i <= num; i++)
        res += dfs(pos + 1, lim && i == num, zero && i == 0);
    return dp[pos][lim][zero] = res;
}
int solve(int x)
{
    len = 0;
    memset(dp, -1, sizeof(dp));
    for (; x; x /= 10)
        a[++len] = x % 10;
    reverse(a + 1, a + len + 1);
    return dfs(1, 1, 1);
}
int main()
{
    int l = read(), r = read();
    printf("%d\n", solve(r) - solve(l - 1));
    return 0;
}
\end{minted}

\hypertarget{ux56feux8bba}{%
\subsection{\# 图论}\label{ux56feux8bba}}

\begin{enumerate}
\def\labelenumi{\arabic{enumi}.}
\tightlist
\item
  dijkstra
\end{enumerate}

\begin{minted}[fontsize=\footnotesize,breaklines,linenos]{cpp}
#include <bits/stdc++.h>
using namespace std;
#define int long long
const int inf = 0x3f3f3f3f;
typedef pair<int,int> PII;
vector<PII> mp[100100];
int n,m,s;
int dis[100100];
int vis[100100];
priority_queue<PII,vector<PII>,greater<PII> > q;

void dj(int s)
{
    for(int i=1;i<=n;i++)dis[i]=inf;
    dis[s]=0ll;
    q.push({dis[s],s});
    
    while(!q.empty())
    {
        int u=q.top().second;
        q.pop();
        if(vis[u])continue;
        vis[u]=1;
        for(auto [w,v]:mp[u])
        {
            if(dis[u]+w<dis[v])
            {
                dis[v]=dis[u]+w;
                q.push({dis[v],v});
            }
        }
    }
}
signed main() {
cin>>n>>m>>s;
for(int i=1;i<=m;i++)
{
    int u,v,w;
    cin>>u>>v>>w;
    mp[u].push_back({w,v});
}
dj(s);
for(int i=1;i<=n;i++)cout<<dis[i]<<" ";
    return 0;
}
\end{minted}

\begin{enumerate}
\def\labelenumi{\arabic{enumi}.}
\setcounter{enumi}{1}
\tightlist
\item
  并查集
\end{enumerate}

\begin{minted}[fontsize=\footnotesize,breaklines,linenos]{cpp}
int fa[100100];
void init(int n)
{
    for(int i=1;i<=n;i++)fa[i]=i;
}
int find(int x)
{
    if(fa[x]==x)
        return x;
    fa[x]=find(fa[x]);
    return find(fa[x]);
}
void merge(int x,int y)
{
    fa[find(x)]=find(y);
}
\end{minted}

\begin{enumerate}
\def\labelenumi{\arabic{enumi}.}
\setcounter{enumi}{2}
\tightlist
\item
  最小生成树 prim
\end{enumerate}

\begin{minted}[fontsize=\footnotesize,breaklines,linenos]{cpp}
   #include <bits/stdc++.h>
using namespace std;
const int N = 5050;
const int inf=0x3f3f;
int g[N][N],dis[200200];
bool vis[200200];
int n,m,ans,u,v,w;
void init(){
    memset(g,inf,sizeof(g));
    memset(dis,inf,sizeof(dis));
}
void addedge(int u,int v,int w)
{
    if (u != v && g[u][v] > w) g[u][v] = g[v][u] = w;
}
void prim(){
    dis[1]=0;
    for(int i=1;i<=n;i++){
        int t=0;
        for(int j=1;j<=n;j++){
            if(!vis[j]&&dis[j]<dis[t])t=j;
        }
        vis[t]=1;
        ans+=dis[t];
        for(int j=1;j<=n;j++){
            if(!vis[j]&&dis[j]>g[t][j]){
                dis[j]=g[t][j];
            }
        }
    }
}

signed main() {
cin>>n>>m;
init();
for(int i=1;i<=m;i++){
    cin>>u>>v>>w;
    addedge(u,v,w);
    addedge(u,v,w);
}
prim();
for(int i=1;i<=n;i++){
    if(!vis[i]) {
        cout << "orz";
        return 0;
    }
}
cout<<ans;
    return 0;
}
\end{minted}

kruskal

\begin{minted}[fontsize=\footnotesize,breaklines,linenos]{cpp}
#include<bits/stdc++.h>
using namespace std;
#define int long long 
int fa[200100];
struct edge{
    int u,v,w;
}e[200100],mst[200100];
int n,m,k;
int ans;
bool cmp(edge a,edge b)
{
    return a.w<b.w;
}
void init(int n)
{
    for(int i=1;i<=n;i++)
    fa[i]=i;
}
int find(int x) 
{
    if(fa[x]==x)
        return x;
    else{
    fa[x]=find(fa[x]); 
    return find(fa[x]);
    }
}
void merge(int i,int j)
{
    fa[find(i)]=find(j);
}
void kruskal(){
    for(int i=1;i<=m;i++){
        if(find(e[i].u)!=find(e[i].v)){
            k++;
            mst[i].u=e[i].u;
            mst[i].v=e[i].v;
            mst[i].w=e[i].w;
            ans+=e[i].w;
            merge(e[i].u,e[i].v);
        }
    }
}
signed main()
{
 std::ios::sync_with_stdio(false);
    std::cin.tie(0);
cin>>n>>m;
init(n);
for(int i=1;i<=m;i++)cin>>e[i].u>>e[i].v>>e[i].w;
sort(e+1,e+m+1,cmp);
kruskal();
if(k==n-1)cout<<ans;
else cout<<"orz";
    return 0;
}
\end{minted}

\begin{enumerate}
\def\labelenumi{\arabic{enumi}.}
\setcounter{enumi}{3}
\tightlist
\item
  lca
\end{enumerate}

\begin{minted}[fontsize=\footnotesize,breaklines,linenos]{cpp}
   #include <bits/stdc++.h>

using namespace std;
#define IOS ios::sync_with_stdio(false), cin.tie(nullptr), cout.tie(nullptr);
#define int long long
#define ull unsigned long long
#define lowbit(i) ((i) & (-i))
#define ls(p) (p << 1)
#define rs(p) (p << 1 | 1)
#define rep(i, a, b) for (int i = a; i <= b; i++)
#define per(i, a, b) for (int i = a; i >= b; i--)

typedef pair<int, int> PII;
const int mod = 1e9 + 7;
const int inf = 0x3f3f3f3f;
const int N = 5e5 + 200;

int qpow(int a, int n)
{
    int ans = 1;
    while (n)
    {
        if (n & 1)
        {
            ans = ans * a % mod;
        }
        a = a * a % mod;
        n >>= 1;
    }
    return ans;
}

int n, q, root; 
vector<int> mp[N];
int lg2[N];
int dep[N];
int f[N][20];
int vis[N];
void dfs(int u, int fa = 0)
{
    if (vis[u])
        return;
    vis[u] = 1;
    dep[u] = dep[fa] + 1;
    f[u][0] = fa;
    for (int i = 1; i <= lg2[dep[u]]; i++)
    {
        f[u][i] = f[f[u][i - 1]][i - 1];
    }
    for (auto v : mp[u])
    {
        dfs(v, u);
    }
}
int lca(int a, int b)
{
    if (dep[a] > dep[b])
        swap(a, b);
    while (dep[a] != dep[b])
        b = f[b][lg2[dep[b] - dep[a]]];
    if (a == b)
        return a;
    for (int k = lg2[dep[a]]; k >= 0; k--)
    {
        if (f[a][k] != f[b][k])
        {
            a = f[a][k], b = f[b][k];
        }
    }
    return f[a][0];
}
signed main()
{
    IOS cin >> n >> q >> root;
    for (int i = 1; i < n; i++)
    {
        int u, v;
        cin >> u >> v;
        mp[u].push_back(v);
        mp[v].push_back(u);
    }
    for (int i = 2; i <= n; i++)
    {
        lg2[i] = lg2[i / 2] + 1;
    } 
    dfs(root);
    while (q--)
    {
        int u, v;
        cin >> u >> v;
        cout << lca(u, v) << endl;
    }
    return 0;
}
\end{minted}

\hypertarget{ux5b57ux7b26ux4e32}{%
\section{字符串}\label{ux5b57ux7b26ux4e32}}

\begin{center}\rule{0.5\linewidth}{0.5pt}\end{center}

\begin{enumerate}
\def\labelenumi{\arabic{enumi}.}
\tightlist
\item
  KMP
\end{enumerate}

\begin{minted}[fontsize=\footnotesize,breaklines,linenos]{cpp}
int f[N];
void kmp(string s, string p)
{
    p += '@';
    p += s;
    for (int i = 1; i < p.size(); i++)
    {
        int j = f[i - 1];
        while (j && p[j] != p[i])
            j = f[j - 1];
        if (p[j] == p[i])
            f[i] = j + 1;
    }
}
\end{minted}

\begin{enumerate}
\def\labelenumi{\arabic{enumi}.}
\setcounter{enumi}{1}
\tightlist
\item
  Manacher
\end{enumerate}

\begin{minted}[fontsize=\footnotesize,breaklines,linenos]{cpp}
int p[N];
string s0, s = "@#";

int main()
{
    cin >> s0;
    for (auto i : s0)
        s += i, s += "#";#include <cstdio>
using namespace std;
using ll = long long;
// !!! N = n * 2, because you need to insert '#' !!!
const int N = 3e7;
#define min(A, B) ((A > B) ? B : A)
// p[i]: range of the palindrome i-centered. 
int p[N];
// s: the string.
char s[N] = "@#";
// l: length of s.
int l = 2;

int main()
{
    char tmp = getchar();
    while (tmp > 'z' || tmp < 'a')
        tmp = getchar();
    while (tmp <= 'z' && tmp >= 'a')
        s[l++] = tmp, s[l++] = '#', tmp = getchar();
    /*<--- input & preparation --->*/
    int m = 0, r = 0;
    ll ans = 0;
    for (int i = 1; i < l; i++)
    {
        // evaluate p[i]
        if (i <= r)
            p[i] = min(p[m * 2 - i], r - i + 1);
        else
            p[i] = 1;
        // brute force!
        while (s[i - p[i]] == s[i + p[i]])
            ++p[i];
        // maintain m, r
        if (i + p[i] > r)
        {
            r = i + p[i] - 1;
            m = i;
        }
        // find the longest p[i]
        if (p[i] > ans)
            ans = p[i]; 
    }
    printf("%lld", ans - 1);

    return 0;
}
    int m = 0, r = 0;
    ll ans = 0;
    int l = s.size();
    for (int i = 1; i < l; i++)
    {
        if (i <= r)
            p[i] = min(p[m * 2 - i], r - i + 1);
        else
            p[i] = 1;
        while (s[i - p[i]] == s[i + p[i]])
            ++p[i];
        if (i + p[i] > r)
        {
            r = i + p[i] - 1;
            m = i;
        }
        if (p[i] > ans)
            ans = p[i];
    }
}
\end{minted}

\begin{enumerate}
\def\labelenumi{\arabic{enumi}.}
\setcounter{enumi}{2}
\tightlist
\item
  hash
\end{enumerate}

\hypertarget{ux968fux673aux7d20ux6570ux8868}{%
\subsubsection{随机素数表}\label{ux968fux673aux7d20ux6570ux8868}}

42737, 46411, 50101, 52627, 54577, 191677, 194869, 210407, 221831,
241337, 578603, 625409, 713569, 788813, 862481, 2174729, 2326673,
2688877, 2779417, 3133583, 4489747, 6697841, 6791471, 6878533, 7883129,
9124553, 10415371, 11134633, 12214801, 15589333, 17148757, 17997457,
20278487, 27256133, 28678757, 38206199, 41337119, 47422547, 48543479,
52834961, 76993291, 85852231, 95217823, 108755593, 132972461, 171863609,
173629837, 176939899, 207808351, 227218703, 306112619, 311809637,
322711981, 330806107, 345593317, 345887293, 362838523, 373523729,
394207349, 409580177, 437359931, 483577261, 490845269, 512059357,
534387017, 698987533, 764016151, 906097321, 914067307, 954169327

1572869, 3145739, 6291469, 12582917, 25165843, 50331653
(适合哈希的素数)

\begin{minted}[fontsize=\footnotesize,breaklines,linenos]{cpp}
#include <string>
using ull = unsigned long long;
using std::string;
const int mod = 998244353;

int n;
ull Hash[2000200]; // 自然溢出法用unsigned类型
ull RHash[2000200];
ull base[2000200];
void init()
{
    base[0] = 1;
    for (int i = 1; i <= 2000010; i++)
    {
        base[i] = base[i - 1] * 131 % mod;
    }
}
void get_hash(string s)
{
    for (int i = 1; i <= (int)s.size(); i++)
    {
        Hash[i] = Hash[i - 1] * base[1] % mod + s[i - 1];
        Hash[i] %= mod;
    }
}
void get_Rhash(string s)
{
    for (int i = (int)s.size(); i >= 1; i--)
    {
        RHash[s.size() - i + 1] = RHash[s.size() - i] * base[1] % mod + s[i - 1];
        RHash[i] %= mod;
    }
}
ull getR(int l, int r)
{
    if (l > r)
        return 0;
    return (RHash[r] - (RHash[l - 1] * base[r - l + 1]) % mod + mod) % mod;
}
ull get(int l, int r)
{
    if (l > r)
        return 0;
    return (Hash[r] - (Hash[l - 1] * base[r - l + 1]) % mod + mod) % mod;
}
\end{minted}

\begin{enumerate}
\def\labelenumi{\arabic{enumi}.}
\setcounter{enumi}{3}
\tightlist
\item
  字典树
\end{enumerate}

\begin{minted}[fontsize=\footnotesize,breaklines,linenos]{cpp}
#include <string>
using std::string;
/* Last modified: 23/07/03 */
// Trie for string and prefix
class Trie
{

    static const int trie_tot_size = 1e5;
    // trie_node_size: modify if get() is modified.
    static const int trie_node_size = 64;
    int tot = 0;
    // end: reserved for count
    const int end = 63;
    int (*nxt)[trie_node_size];

  public:
    Trie()
    {
        nxt = new (int[trie_tot_size][trie_node_size]);
    }
    int get(char x)
    {
        // modify if x is in certain range, assuming 0-9 or a-z.
        if (x >= 'A' && x <= 'Z')
            return x - 'A';
        else if (x >= 'a' && x <= 'z')
            return x - 'a' + 26;
        else
            return x - '0' + 52;
    }
    int find(string s)
    {
        int cnt = 0;
        for (auto i : s)
        {
            cnt = nxt[cnt][get(i)];
            if (!cnt)
                return 0;
        }
        return cnt;
    }
    void insert(string s)
    {
        int cnt = 0;
        for (auto i : s)
        {
            auto j = get(i);
            // count how many strings went by
            nxt[cnt][end]++;
            if (nxt[cnt][j] > 0)
                // character i already exists.
                cnt = nxt[cnt][j];
            else
            {
                // doesn't exist, new node.
                nxt[cnt][j] = ++tot;
                cnt = tot;
            }
        }
        nxt[cnt][end]++;
    }
    int count(string s)
    {
        int cnt = find(s);
        if (!cnt)
            return 0;
        return nxt[cnt][end];
    }
    void clear()
    {
        for (int i = 0; i <= tot; i++)
            for (int j = 0; j <= end; j++)
                nxt[i][j] = 0;
        tot = 0;
    }
};
\end{minted}

5.AC自动机

\begin{minted}[fontsize=\footnotesize,breaklines,linenos]{cpp}
#include <queue>
#include <string.h>
#include <string>
using std::string, std::queue;

struct AC_automaton
{
    static const int _N = 1e6;

    int (*trie)[27];
    int tot = 0;
    int *fail;
    int *e;
    AC_automaton()
    {
        trie = new int[_N][27];
        fail = new int[_N];
        e = new int[_N];
        memset(trie, 0, sizeof(trie));
        memset(fail, 0, sizeof(fail));
        memset(e, 0, sizeof(e));
    }

    void insert(string s)
    {
        int now = 0;
        for (auto i : s)
            if (trie[now][i - 'a'])
                now = trie[now][i - 'a'];
            else
                trie[now][i - 'a'] = ++tot, now = tot;
        e[now]++;
    }

    void build()
    {
        queue<int> q;
        for (int i = 0; i < 26; i++)
            if (trie[0][i])
                q.push(trie[0][i]);
        while (q.size())
        {
            int u = q.front();
            q.pop();
            for (int i = 0; i < 26; i++)
                if (trie[u][i])
                    fail[trie[u][i]] = trie[fail[u]][i], q.push(trie[u][i]);
                else
                    trie[u][i] = trie[fail[u]][i];
        }
    }

    int query(string t)
    {
        int u = 0, res = 0;
        for (auto c : t)
        {
            u = trie[u][c - 'a'];
            for (int j = u; j && e[j] != -1; j = fail[j])
                res += e[j], e[j] = -1;
        }
        return res;
    }
};

\end{minted}

\hypertarget{ux5fc3ux6001ux5d29ux4e86}{%
\section{心态崩了}\label{ux5fc3ux6001ux5d29ux4e86}}

\begin{itemize}
\tightlist
\item
  \texttt{(int)v.size()}
\item
  \texttt{1LL\ \textless{}\textless{}\ k}
\item
  递归函数用全局或者 static 变量要小心
\item
  预处理组合数注意上限
\item
  想清楚到底是要 \texttt{multiset} 还是 \texttt{set}
\item
  提交之前看一下数据范围,测一下边界
\item
  数据结构注意数组大小 (2倍,4倍)
\item
  字符串注意字符集
\item
  如果函数中使用了默认参数的话,注意调用时的参数个数。
\item
  注意要读完
\item
  构造参数无法使用自己
\item
  树链剖分/dfs 序,初始化或者询问不要忘记 idx, ridx
\item
  排序时注意结构体的所有属性是不是考虑了
\item
  不要把 while 写成 if
\item
  不要把 int 开成 char
\item
  清零的时候全部用 0\textasciitilde n+1。
\item
  模意义下不要用除法
\item
  哈希不要自然溢出
\item
  最短路不要 SPFA,乖乖写 Dijkstra
\item
  上取整以及 GCD 小心负数
\item
  mid 用 \texttt{l\ +\ (r\ -\ l)\ /\ 2} 可以避免溢出和负数的问题
\item
  小心模板自带的意料之外的隐式类型转换
\item
  求最优解时不要忘记更新当前最优解
\item
  图论问题一定要注意图不连通的问题
\item
  处理强制在线的时候 lastans 负数也要记得矫正
\item
  不要觉得编译器什么都能优化
\item
  分块一定要特判在同一块中的情况
\end{itemize}

\hypertarget{ux5febux8bfb}{%
\section{快读}\label{ux5febux8bfb}}

\begin{center}\rule{0.5\linewidth}{0.5pt}\end{center}

\begin{minted}[fontsize=\footnotesize,breaklines,linenos]{cpp}
inline int read()
{
    int x = 0, w = 1;
    char ch = 0;
    while (ch < '0' || ch > '9')
    {
        ch = getchar();
        if (ch == '-')
        {
            w = -1;
        }
    }
    while (ch >= '0' && ch <= '9')
    {
        x = x * 10 + ch - '0';
        ch = getchar();
    }
    return x * w;
}
\end{minted}

\hypertarget{ux6570ux5b66}{%
\subsection{\# 数学}\label{ux6570ux5b66}}

\begin{enumerate}
\def\labelenumi{\arabic{enumi}.}
\tightlist
\item
  快速幂
\end{enumerate}

\begin{minted}[fontsize=\footnotesize,breaklines,linenos]{cpp}
int qpow(int a,int n)
{
    int ans=1;
    while(n)
    {
        if(n&1)
        {
            ans =ans*a%mod;
        }
        a=a*a%mod;
        n>>=1;

    }
    return ans;
}
\end{minted}

\begin{enumerate}
\def\labelenumi{\arabic{enumi}.}
\setcounter{enumi}{1}
\tightlist
\item
  exgcd
\end{enumerate}

\begin{minted}[fontsize=\footnotesize,breaklines,linenos]{cpp}
int exgcd(int a,int b,int &x,int &y)
{
    if(b==0)
    {
        x=1;
        y=0;
        return a;
    }
    int d=exgcd(b,a%b,x,y),x0=x,y0=y;
    x=y0;
    y=x0-(a/b)*y0;
    return d;
}
\end{minted}

\begin{enumerate}
\def\labelenumi{\arabic{enumi}.}
\setcounter{enumi}{2}
\tightlist
\item
  线性inv
\end{enumerate}

\begin{minted}[fontsize=\footnotesize,breaklines,linenos]{cpp}
void getinv(int n)
{
    inv[1]=1;
    for(int i=2;i<=n;i++)
    {
        inv[i]=mod-((mod/i)*inv[mod%i])%mod;
    }
}
\end{minted}

\begin{enumerate}
\def\labelenumi{\arabic{enumi}.}
\setcounter{enumi}{3}
\tightlist
\item
  分块
\end{enumerate}

\begin{minted}[fontsize=\footnotesize,breaklines,linenos]{cpp}
int ans=0;
    for(int l=1,r;l<=n;l=r+1)
    {
        r=n/(n/l);
        ans+=(r-l+1)*(n/l);
    }
    cout<<ans<<endl;
\end{minted}

\begin{enumerate}
\def\labelenumi{\arabic{enumi}.}
\setcounter{enumi}{4}
\tightlist
\item
  欧拉筛
\end{enumerate}

\begin{minted}[fontsize=\footnotesize,breaklines,linenos]{cpp}
int Eular(int n) {
    int cnt = 0;
    memset(is_prime, true, sizeof(is_prime));
    is_prime[0] = is_prime[1] = false;
    for(int i=2;i<=n;i++)
    {
        if(is_prime[i])
        {
            prime[++cnt]=i;
        }
        for(int j=1;j<=cnt&&i*prime[j]<=n;j++)
        {
            is_prime[i*prime[j]]=0;
            if(i%prime[j]==0)break;
        }
    }
    return cnt;
}
\end{minted}

\begin{enumerate}
\def\labelenumi{\arabic{enumi}.}
\setcounter{enumi}{5}
\tightlist
\item
  欧拉函数
\end{enumerate}

\begin{minted}[fontsize=\footnotesize,breaklines,linenos]{cpp}
int Eular(int n) {
    int cnt = 0;
    memset(is_prime, true, sizeof(is_prime));
    is_prime[0] = is_prime[1] = false;
    for(int i=2;i<=n;i++)
    {
        if(is_prime[i])
        {
            prime[++cnt]=i;
        }
        for(int j=1;j<=cnt&&i*prime[j]<=n;j++)
        {
            is_prime[i*prime[j]]=0;
            if(i%prime[j]==0)break;
        }
    }
    return cnt;
}
\end{minted}

\begin{enumerate}
\def\labelenumi{\arabic{enumi}.}
\setcounter{enumi}{6}
\tightlist
\item
  组合数
\end{enumerate}

\begin{minted}[fontsize=\footnotesize,breaklines,linenos]{cpp}
int fac[N];
int  inv[N];
void init(int n)
{
    fac[0] = 1;
    inv[0] = 1;
    inv[1] = 1;
    fac[1] = 1;
    for(int i = 2;i<=2*n;i++)
    {
        fac[i] = fac[i-1]*i%mod;
        inv[i] = (mod-mod/i)*inv[mod%i]%mod;
    }
    for(int i = 1;i<=n;i++)
    {
        inv[i] = inv[i]*inv[i-1]%mod;
    }
}
int C(int n,int m)
{
    if(m>n||m<0||n<0)return 0;
    return fac[n]*inv[m]%mod*inv[n-m]%mod;
}
\end{minted}

\begin{enumerate}
\def\labelenumi{\arabic{enumi}.}
\setcounter{enumi}{7}
\item
  欧拉定理 \(a\) 在 \(\bmod m\)意义下, \(a(bc)\) 与
  \(a ^ (b \bmod(eular(m))+m)\) 同余
\item
  卡特兰数
\end{enumerate}

\begin{minted}[fontsize=\footnotesize,breaklines,linenos]{cpp}
int C(int n,int m)
{
    return fac[n]*qpow(fac[n-m],mod-2)%mod*qpow(fac[m],mod-2)%mod;
}
int cat(int n)
{
    return C(2*n,n)*qpow(n+1,mod-2)%mod;
}
\end{minted}

\begin{enumerate}
\def\labelenumi{\arabic{enumi}.}
\setcounter{enumi}{9}
\tightlist
\item
  矩阵
\end{enumerate}

\begin{minted}[fontsize=\footnotesize,breaklines,linenos]{cpp}
    class matrix
{
    public:
        int x[105][105];
        int sz;
        matrix(int n)
        {sz=n;
            for(int i=1;i<=sz;i++)
            {
                for(int j=1;j<=sz;j++)
                {
                    x[i][j]=0;
                }
            }
        }
        matrix mul(matrix a,matrix b);
        matrix qpow(matrix a,int n);
        void tra(matrix a);
};

matrix matrix::mul(matrix a, matrix b) {
    matrix c(a.sz);
    for(int i=1;i<=a.sz;i++)
        for(int j=1;j<=a.sz;j++)
            for(int k=1;k<=a.sz;k++)
                c.x[i][j]=(c.x[i][j]%mod+(a.x[i][k]*b.x[k][j])%mod)%mod;
    return c;
}
matrix matrix::qpow(matrix a,int n)
{
    matrix res(a.sz);
    for(int i=1;i<=a.sz;i++)res.x[i][i]=1;
    while(n>0)
    {
        if(n&1)res= mul(res,a);
        a= mul(a,a);
        n>>=1;
    }
    return res;
}
void matrix::tra(matrix a) {
    for(int i=1;i<=a.sz;i++)
    {
        for(int j=1;j<=a.sz;j++)
        {
            cout<<a.x[i][j]<<" ";
        }
        cout<<endl;
    }
}
\end{minted}

\hypertarget{ux6570ux636eux7ed3ux6784}{%
\section{数据结构}\label{ux6570ux636eux7ed3ux6784}}

\begin{center}\rule{0.5\linewidth}{0.5pt}\end{center}

1.树状数组

\begin{minted}[fontsize=\footnotesize,breaklines,linenos]{cpp}
class BIT
{
    int n = 2e6;
    long long *a;

  public:
    BIT(int size) : n(size)
    {
        a = new long long[size + 10];
    }
    void update(int p, long long x)
    {
        while (p <= n)
            a[p] += x, p += (p & (-p));
    }

    long long query(int l, int r)
    {
        long long ret = 0;
        l--;
        while (r > 0)
            ret += a[r], r -= (r & (-r));
        while (l > 0)
            ret -= a[l], l -= (l & (-l));
        return ret;
    }
};

\end{minted}

\begin{enumerate}
\def\labelenumi{\arabic{enumi}.}
\setcounter{enumi}{1}
\tightlist
\item
  并查集
\end{enumerate}

\begin{minted}[fontsize=\footnotesize,breaklines,linenos]{cpp}
#include <cassert>
#include <iostream>
#include <set>
/* Last modified: 23/08/01 */
class DSU
{
  private:
    int *f;
    int size;

  public:
    DSU(int size) : size(size)
    {
        assert(size > 1);
        f = new int[size + 10];
        for (int i = 1; i <= size; i++)
            f[i] = i;
    }
    int find(int x)
    {
        return f[x] == x ? x : (f[x] = find(f[x]));
    };
    bool same(int x, int y)
    {
        return find(x) == find(y);
    };
    bool merge(int x, int y)
    {
        int fx = find(x), fy = find(y);
        return ((fx != fy) ? f[fx] = fy : false);
    };
    int count()
    {
        std::set<int> s;
        for (int i = 1; i <= size; i++)
            s.insert(find(i));
        return s.size();
    }
};
\end{minted}

\begin{enumerate}
\def\labelenumi{\arabic{enumi}.}
\setcounter{enumi}{2}
\tightlist
\item
  ST表
\end{enumerate}

\begin{minted}[fontsize=\footnotesize,breaklines,linenos]{cpp}
// log2(x) 的预处理
// 1. 递推
lg[2] = 1;
for (int i = 3; i < N; i++)
    lg[i] = lg[i / 2] + 1;
// 2. 基于编译期计算
using std::array;
// WARNING: LOG_SIZE may cause CE if too big.
const int LOG_SIZE = 1e5 + 10;
constexpr array<int, LOG_SIZE> LOG = []() {
    array<int, LOG_SIZE> l{0, 0, 1};
    for (int i = 3; i < LOG_SIZE; i++)
        l[i] = l[i / 2] + 1;
    return l;
}();
// 3. 直接计算
int lg(int x)
{
    return 31 - __builtin_clz(x);
}
// STL 提供了 std::lg(), 底数是e.
\end{minted}

\begin{minted}[fontsize=\footnotesize,breaklines,linenos]{cpp}
class SparseTable
{
  private:
    // SIZE depends on range of f[i][0].
    // 22 is suitable for 1e5.
    static const int SIZE = 22;
    // f[i][j] maintains the result from i to i + 2 ^ j - 1;
    int (*f)[SIZE];
    using func = std::function<int(int, int)>;
    func op;
    // length of f from 1 to l;
    int l;

  public:
    SparseTable(int a[][SIZE], func foo, int len) : f(a), op(foo), l(len)
    {
        for (int j = 1; j < SIZE; j++)
            for (int i = 1; i + (1 << j) - 1 <= len; i++)
                // f[i][j] comes from f[i][j - 1].
                // f[i][j - 1], f[i + 2^(j - 1)] cover the range of f[i][j].
                f[i][j] = foo(f[i][j - 1], f[i + (1 << (j - 1))][j - 1]);
    };
    int query(int x, int y)
    {
        int s = LOG[y - x + 1];
        return op(f[x][s], f[y - (1 << s) + 1][s]);
    }
};
\end{minted}

\begin{enumerate}
\def\labelenumi{\arabic{enumi}.}
\setcounter{enumi}{3}
\tightlist
\item
  线段树
\end{enumerate}

\begin{minted}[fontsize=\footnotesize,breaklines,linenos]{cpp}
#include <bits/stdc++.h>

using namespace std;
#define IOS ios::sync_with_stdio(false), cin.tie(nullptr), cout.tie(nullptr);
#define int long long
#define ull unsigned long long
#define lowbit(i) ((i) & (-i))
#define ls(p) (p << 1)
#define rs(p) (p << 1 | 1)
#define rep(i, a, b) for (int i = a; i <= b; i++)
#define per(i, a, b) for (int i = a, i >= b, i--)

typedef pair<int, int> PII;
const int mod = 1e9 + 7;
const int inf = 0x3f3f3f3f;
const int N = 1e5 + 200;
int qpow(int a, int n)
{
    int ans = 1;
    while (n)
    {
        if (n & 1)
        {
            ans = ans * a % mod;
        }
        a = a * a % mod;
        n >>= 1;
    }
    return ans;
}
int a[N];
int tag[4 * N];
int tree[4 * N];
int n;
void push_up(int p)
{
    tree[p] = tree[ls(p)] + tree[rs(p)];
}
void build(int p, int l, int r)
{
    if (l == r)
    {
        tree[p] = a[l];
        return;
    }
    int mid = (l + r) >> 1;
    build(ls(p), l, mid);
    build(rs(p), mid + 1, r);
    push_up(p);
}
void push_down(int p, int l, int r)
{
    int mid = (l + r) >> 1;
    tag[ls(p)] += tag[p];
    tag[rs(p)] += tag[p];
    tree[ls(p)] += tag[p] * (mid - l + 1);
    tree[rs(p)] += tag[p] * (r - mid);
    tag[p] = 0;
}
void update(int nl, int nr, int k, int p = 1, int l = 1, int r = n)
{
    if (nl <= l && r <= nr)
    {
        tag[p] += k;
        tree[p] += k * (r - l + 1);
        return;
    }
    push_down(p, l, r);
    int mid = (l + r) >> 1;
    if (nl <= mid)
        update(nl, nr, k, ls(p), l, mid);
    if (nr > mid)
        update(nl, nr, k, rs(p), mid + 1, r);
    push_up(p);
}
int query(int x, int y, int l = 1, int r = n, int p = 1)
{
    int res = 0;
    if (x <= l && y >= r)
        return tree[p];
    int mid = (l + r) >> 1;
    push_down(p, l, r);
    if (x <= mid)
        res += query(x, y, l, mid, ls(p));
    if (y > mid)
        res += query(x, y, mid + 1, r, rs(p));
    return res;
}
signed main()
{
    IOS int q;
    cin >> n >> q;
    for (int i = 1; i <= n; i++)
        cin >> a[i];
    build(1, 1, n);
    while (q--)
    {
        int op, x, y, k;
        cin >> op;
        if (op == 1)
        {
            cin >> x >> y >> k;
            update(x, y, k);
        }
        else
        {
            cin >> x >> y;
            cout << query(x, y) << endl;
        }
    }
    return 0;
}
\end{minted}

\end{document}
